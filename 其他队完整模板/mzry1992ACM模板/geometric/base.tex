\subsection{基本函数}
    \subsubsection{Point定义}
        \begin{lstlisting}[language=c++]
struct Point
{
    double x, y;
    Point() {}
    Point(double _x, double _y)
    {
        x = _x, y = _y;
    }
    Point operator -(const Point &b)const
    {
        return Point(x - b.x, y - b.y);
    }
    double operator *(const Point &b)const
    {
        return x * b.y - y * b.x;
    }
    double operator &(const Point &b)const
    {
        return x * b.x + y * b.y;
    }
};
        \end{lstlisting}
    \subsubsection{Line定义}
        \begin{lstlisting}[language=c++]
struct Line
{
    Point s, e;
    double k;
    Line() {}
    Line(Point _s, Point _e)
    {
        s = _s, e = _e;
        k = atan2(e.y - s.y, e.x - s.x);
    }
    Point operator &(const Line &b)const
    {
        Point res = s;
        //注意:有些题目可能会有直线相交或者重合情况
        //可以把返回值改成`pair<Point,int>`来返回两直线的状态。
        double t = ((s - b.s) * (b.s - b.e)) / ((s - e) * (b.s - b.e));
        res.x += (e.x - s.x) * t;
        res.y += (e.y - s.y) * t;
        return res;
    }
};
        \end{lstlisting}
    \subsubsection{距离:两点距离}
        \begin{lstlisting}[language=c++]
double dist2(Point a, Point b)
{
    return (a.x - b.x) * (a.x - b.x) + (a.y - b.y) * (a.y - b.y);
}
        \end{lstlisting}
    \subsubsection{距离:点到线段距离}
        res:点到线段最近点
        \begin{lstlisting}[language=c++]
double dist2(Point p1, Point p2, Point p)
{
    Point res;
    double a, b, t;
    a = p2.x - p1.x;
    b = p2.y - p1.y;
    t = ((p.x - p1.x) * a + (p.y - p1.y) * b) / (a * a + b * b);
    if (t >= 0 && t <= 1)
    {
        res.x = p1.x + a * t;
        res.y = p1.y + b * t;
    }
    else
    {
        if (dist2(p, p1) < dist2(p, p2))
            res = p1;
        else
            res = p2;
    }
    return dist2(p, res);
}
        \end{lstlisting}
    \subsubsection{面积:多边形}
        点按逆时针排序。
        \begin{lstlisting}[language=c++]
double CalcArea(Point p[], int n)
{
    double res = 0;
    for (int i = 0; i < n; i++)
        res += (p[i] * p[(i + 1) % n]) / 2;
    return res;
}
        \end{lstlisting}
    \subsubsection{判断:线段相交}
        \begin{lstlisting}[language=c++]
bool inter(Line l1,Line l2)
{
    return (max(l1.s.x,l1.e.x) >= min(l2.s.x,l2.e.x) &&
            max(l2.s.x,l2.e.x) >= min(l1.s.x,l1.e.x) &&
            max(l1.s.y,l1.e.y) >= min(l2.s.y,l2.e.y) &&
            max(l2.s.y,l2.e.y) >= min(l1.s.y,l1.e.y) &&
            ((l2.s-l1.s)*(l1.e-l1.s))*((l2.e-l1.s)*(l1.e-l1.s)) <= 0 &&			//`若端点不算相交则两个<=改为<即可`
            ((l1.s-l2.s)*(l2.e-l2.s))*((l1.e-l2.s)*(l2.e-l2.s)) <= 0);
}
        \end{lstlisting}
    \subsubsection{求解:点到线最近点}
	\begin{lstlisting}[language=c++]
Point NPT(Point P, Line L)
{
    Point result;
    double a, b, t;
    a = L.e.x - L.s.x;
    b = L.e.y - L.s.y;
    t = ((P.x - L.s.x) * a + (P.y - L.s.y) * b ) / (a * a + b * b);
    //`如果t小于0或者大于1,说明最近点在L.s和L.e这条线段之外`
    result.x = L.s.x + a * t;
    result.y = L.s.y + b * t;
    return result;
}
	\end{lstlisting}