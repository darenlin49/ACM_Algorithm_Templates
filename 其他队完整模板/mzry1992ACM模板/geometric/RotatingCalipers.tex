\subsection{旋转卡壳}
    “对踵”\\
    \subsubsection{单个凸包}
	\begin{lstlisting}[language=c++]
void solve(Point p[],int n)
{
    Point v;
    int cur = 1;
    for (int i = 0;i < n;i++)
    {
        v = p[i]-p[(i+1)%n];
        while (v*(p[(cur+1)%n]-p[cur]) < 0)
            cur = (cur+1)%n;
        //p[cur] -> p[i]
        //p[cur] -> p[i+1]
        //p[cur] -> (p[i],p[i+1])
    }
}
	\end{lstlisting}
    \subsubsection{两个凸包}
	注意初始点的选取,代码只是个示例。\\
	有时候答案需要取solve(p0,n,p1,m)和solve(p1,m,p0,n)的最优值。\\
	\begin{lstlisting}[language=c++]
void solve(Point p0[],int n,Point p1[],int m)
{
    Point v;
    int cur = 0;
    for (int i = 0;i < n;i++)
    {
        v = p0[i]-p0[(i+1)%n];
        while (v*(p1[(cur+1)%m]-p1[cur]) < 0)
            cur = (cur+1)%m;
        //p1[cur] -> p0[i]
        //p1[cur] -> p0[i+1]
        //p1[cur] -> (p0[i],p0[i+1])
    }
}
	\end{lstlisting}
    \subsubsection{外接矩形}
	\begin{lstlisting}[language=c++]
void solve()
{
    resa = resb = 1e100;
    double dis1,dis2;
    Point xp[4];
    Line l[4];
    int a,b,c,d;
    int sa,sb,sc,sd;
    a = b = c = d = 0;
    sa = sb = sc = sd = 0;
    Point va,vb,vc,vd;
    for (a = 0; a < n; a++)
    {
        va = Point(p[a],p[(a+1)%n]);
        vc = Point(-va.x,-va.y);
        vb = Point(-va.y,va.x);
        vd = Point(-vb.x,-vb.y);
        if (sb < sa)
        {
            b = a;
            sb = sa;
        }
        while (xmult(vb,Point(p[b],p[(b+1)%n])) < 0)
        {
            b = (b+1)%n;
            sb++;
        }
        if (sc < sb)
        {
            c = b;
            sc = sb;
        }
        while (xmult(vc,Point(p[c],p[(c+1)%n])) < 0)
        {
            c = (c+1)%n;
            sc++;
        }
        if (sd < sc)
        {
            d = c;
            sd = sc;
        }
        while (xmult(vd,Point(p[d],p[(d+1)%n])) < 0)
        {
            d = (d+1)%n;
            sd++;
        }

        //卡在p[a],p[b],p[c],p[d]上
        sa++;
    }
}
	\end{lstlisting}